<<<<<<< HEAD
Inicia-se a descrição deste perímetro no vértice DBT-M-5059, 
com Latitude -14°50'49,334" N e Longitude -42°34'58,808" S, deste, segue confrontando com ANAHI DO CARMO CERQUEIRA SANTANA, com azimute geodésico de 193°26' e distância de 
60,27m até o vértice DBT-M-5058 de Latitude -14°50'49,334" N e Logitude -42°34'58,808" S;
deste, segue confrontando com ANAHI DO CARMO CERQUEIRA SANTANA, com azimute geodésico de 183°42' e distância de 
53,91m até o vértice DBT-M-5057 de Latitude -14°50'51,241" N e Logitude -42°34'59,277" S;
deste, segue confrontando com ANAHI DO CARMO CERQUEIRA SANTANA, com azimute geodésico de 170°46' e distância de 
48,65m até o vértice DBT-M-5056 de Latitude -14°50'52,991" N e Logitude -42°34'59,393" S;
deste, segue confrontando com WELISMAR BORBOREMA CARVALHO, com azimute geodésico de 205°50' e distância de 
559,05m até o vértice DBT-M-5172 de Latitude -14°50'54,554" N e Logitude -42°34'59,132" S;
deste, segue confrontando com SEBASTIÃO SOUZA E SILVA, com azimute geodésico de 306°29' e distância de 
460,93m até o vértice DBT-M-A053 de Latitude -14°51'10,922" N e Logitude -42°35'07,281" S;
deste, segue confrontando com SEBASTIÃO SOUZA E SILVA, com azimute geodésico de 243°25' e distância de 
721,29m até o vértice DBT-M-A052 de Latitude -14°51'02,004" N e Logitude -42°35'19,673" S;
deste, segue confrontando com SEBASTIÃO SOUZA E SILVA, com azimute geodésico de 222°45' e distância de 
396,7m até o vértice DBT-M-A051 de Latitude -14°51'12,502" N e Logitude -42°35'41,247" S;
deste, segue confrontando com RIO AGUA BRANCA, com azimute geodésico de 271°31' e distância de 
39,57m até o vértice DBT-P-1006101 de Latitude -14°51'21,976" N e Logitude -42°35'50,256" S;
deste, segue confrontando com RIO AGUA BRANCA, com azimute geodésico de 301°01' e distância de 
48,32m até o vértice DBT-P-1006102 de Latitude -14°51'21,941" N e Logitude -42°35'51,579" S;
deste, segue confrontando com RIO AGUA BRANCA, com azimute geodésico de 308°29' e distância de 
37,21m até o vértice DBT-P-1006103 de Latitude -14°51'21,131" N e Logitude -42°35'52,964" S;
deste, segue confrontando com RIO AGUA BRANCA, com azimute geodésico de 260°24' e distância de 
39,83m até o vértice DBT-V-1000102 de Latitude -14°51'20,378" N e Logitude -42°35'53,938" S;
deste, segue confrontando com RIO AGUA BRANCA, com azimute geodésico de 310°21' e distância de 
28,85m até o vértice DBT-V-1000103 de Latitude -14°51'20,593" N e Logitude -42°35'55,251" S;
deste, segue confrontando com RIO AGUA BRANCA, com azimute geodésico de 292°52' e distância de 
58,18m até o vértice DBT-V-1000104 de Latitude -14°51'19,986" N e Logitude -42°35'55,986" S;
deste, segue confrontando com RIO AGUA BRANCA, com azimute geodésico de 283°24' e distância de 
112,54m até o vértice DBT-V-1000105 de Latitude -14°51'19,250" N e Logitude -42°35'57,779" S;
deste, segue confrontando com RIO AGUA BRANCA, com azimute geodésico de 292°53' e distância de 
83,63m até o vértice DBT-V-1000106 de Latitude -14°51'18,401" N e Logitude -42°36'01,441" S;
deste, segue confrontando com RIO AGUA BRANCA, com azimute geodésico de 269°14' e distância de 
53,86m até o vértice DBT-V-1000107 de Latitude -14°51'17,342" N e Logitude -42°36'04,018" S;
deste, segue confrontando com CNS: 01.017-3 - Mat.:882, com azimute geodésico de 25°29' e distância de 
73,13m até o vértice DBT-M-A036 de Latitude -14°51'17,366" N e Logitude -42°36'05,819" S;
deste, segue confrontando com CNS: 01.017-3 - Mat.:882, com azimute geodésico de 39°01' e distância de 
518,41m até o vértice DBT-M-A037 de Latitude -14°51'15,218" N e Logitude -42°36'04,767" S;
deste, segue confrontando com CNS: 01.017-3 - Mat.:3217, com azimute geodésico de 76°34' e distância de 
1692,03m até o vértice DBT-M-5059 de Latitude -14°51'02,117" N e Logitude -42°35'53,849" S;
 encerrando esta descrição. Todas as coordenadas aqui descritas estão georrefereciadas ao Sistema 
Geodésico Brasileiro, e encontram-se representadas no sistema SIRGAS 2000, referenciadas ao Meridiano Central -39°.
 Todos os azimutes e distâncias, área e perímetro foram calculados no Sistema Geodésico Local.
=======
Inicia-se a descrição deste perímetro no vértice DBT-M-5059, com Latitude -14°50'49,334" N e Longitude -42°34'58,808" S; deste, segue confrontando com ANAHI DO CARMO CERQUEIRA SANTANA, com azimute geodésico de 193°26' e distância de 60,27m até o vértice DBT-M-5058, com Latitude -14°50'51,241" N e Longitude -42°34'59,277" S; deste, segue confrontando com ANAHI DO CARMO CERQUEIRA SANTANA, com azimute geodésico de 183°42' e distância de 
53,91m  até o vértice DBT-M-5057, com Latitude -14°50'52,991" N e Longitude -42°34'59,393" S; deste, segue confrontando com ANAHI DO CARMO CERQUEIRA SANTANA, com azimute geodésico de 170°46' e distância de 
48,65m  até o vértice DBT-M-5056, com Latitude -14°50'54,554" N e Longitude -42°34'59,132" S; deste, segue confrontando com WELISMAR BORBOREMA CARVALHO, com azimute geodésico de 205°50' e distância de 
559,05m  até o vértice DBT-M-5172, com Latitude -14°51'10,922" N e Longitude -42°35'07,281" S; deste, segue confrontando com SEBASTIÃO SOUZA E SILVA, com azimute geodésico de 306°29' e distância de 
460,93m  até o vértice DBT-M-A053, com Latitude -14°51'02,004" N e Longitude -42°35'19,673" S; deste, segue confrontando com SEBASTIÃO SOUZA E SILVA, com azimute geodésico de 243°25' e distância de 
721,29m  até o vértice DBT-M-A052, com Latitude -14°51'12,502" N e Longitude -42°35'41,247" S; deste, segue confrontando com SEBASTIÃO SOUZA E SILVA, com azimute geodésico de 222°45' e distância de 
396,7m  até o vértice DBT-M-A051, com Latitude -14°51'21,976" N e Longitude -42°35'50,256" S; deste, segue confrontando com RIO AGUA BRANCA, com azimute geodésico de 271°31' e distância de 
39,57m  até o vértice DBT-P-1006101, com Latitude -14°51'21,941" N e Longitude -42°35'51,579" S; deste, segue confrontando com RIO AGUA BRANCA, com azimute geodésico de 301°01' e distância de 
48,32m  até o vértice DBT-P-1006102, com Latitude -14°51'21,131" N e Longitude -42°35'52,964" S; deste, segue confrontando com RIO AGUA BRANCA, com azimute geodésico de 308°29' e distância de 
37,21m  até o vértice DBT-P-1006103, com Latitude -14°51'20,378" N e Longitude -42°35'53,938" S; deste, segue confrontando com RIO AGUA BRANCA, com azimute geodésico de 260°24' e distância de 
39,83m  até o vértice DBT-V-1000102, com Latitude -14°51'20,593" N e Longitude -42°35'55,251" S; deste, segue confrontando com RIO AGUA BRANCA, com azimute geodésico de 310°21' e distância de 
28,85m  até o vértice DBT-V-1000103, com Latitude -14°51'19,986" N e Longitude -42°35'55,986" S; deste, segue confrontando com RIO AGUA BRANCA, com azimute geodésico de 292°52' e distância de 
58,18m  até o vértice DBT-V-1000104, com Latitude -14°51'19,250" N e Longitude -42°35'57,779" S; deste, segue confrontando com RIO AGUA BRANCA, com azimute geodésico de 283°24' e distância de 
112,54m  até o vértice DBT-V-1000105, com Latitude -14°51'18,401" N e Longitude -42°36'01,441" S; deste, segue confrontando com RIO AGUA BRANCA, com azimute geodésico de 292°53' e distância de 
83,63m  até o vértice DBT-V-1000106, com Latitude -14°51'17,342" N e Longitude -42°36'04,018" S; deste, segue confrontando com RIO AGUA BRANCA, com azimute geodésico de 269°14' e distância de 
53,86m  até o vértice DBT-V-1000107, com Latitude -14°51'17,366" N e Longitude -42°36'05,819" S; deste, segue confrontando com CNS: 01.017-3 - Mat.:882, com azimute geodésico de 25°29' e distância de 
73,13m  até o vértice DBT-M-A036, com Latitude -14°51'15,218" N e Longitude -42°36'04,767" S; deste, segue confrontando com CNS: 01.017-3 - Mat.:882, com azimute geodésico de 39°01' e distância de 
518,41m  até o vértice DBT-M-A037, com Latitude -14°51'02,117" N e Longitude -42°35'53,849" S; deste, segue confrontando com CNS: 01.017-3 - Mat.:3217, com azimute geodésico de 76°34' e distância de 
1692,03m  até o vértice DBT-M-5059 encerrando esta descrição. Todas as coordenadas aqui descritas estão georrefereciadas ao Sistema Geodésico Brasileiro, e encontram-se representadas no sistema SIRGAS 2000, referenciadas ao Meridiano Central -39°. Todos os azimutes e distâncias, área e perímetro foram calculados no Sistema Geodésico Local.Inicia-se a descrição deste perímetro no vértice DBT-M-5059, com Latitude -14°50'49,334" N e Longitude -42°34'58,808" S; deste, segue confrontando com ANAHI DO CARMO CERQUEIRA SANTANA, com azimute geodésico de 193°26' e distância de 60,27m até o vértice DBT-M-5058, com Latitude -14°50'51,241" N e Longitude -42°34'59,277" S; deste, segue confrontando com ANAHI DO CARMO CERQUEIRA SANTANA, com azimute geodésico de 183°42' e distância de 
53,91m  até o vértice DBT-M-5057, com Latitude -14°50'52,991" N e Longitude -42°34'59,393" S; deste, segue confrontando com ANAHI DO CARMO CERQUEIRA SANTANA, com azimute geodésico de 170°46' e distância de 
48,65m  até o vértice DBT-M-5056, com Latitude -14°50'54,554" N e Longitude -42°34'59,132" S; deste, segue confrontando com WELISMAR BORBOREMA CARVALHO, com azimute geodésico de 205°50' e distância de 
559,05m  até o vértice DBT-M-5172, com Latitude -14°51'10,922" N e Longitude -42°35'07,281" S; deste, segue confrontando com SEBASTIÃO SOUZA E SILVA, com azimute geodésico de 306°29' e distância de 
460,93m  até o vértice DBT-M-A053, com Latitude -14°51'02,004" N e Longitude -42°35'19,673" S; deste, segue confrontando com SEBASTIÃO SOUZA E SILVA, com azimute geodésico de 243°25' e distância de 
721,29m  até o vértice DBT-M-A052, com Latitude -14°51'12,502" N e Longitude -42°35'41,247" S; deste, segue confrontando com SEBASTIÃO SOUZA E SILVA, com azimute geodésico de 222°45' e distância de 
396,7m  até o vértice DBT-M-A051, com Latitude -14°51'21,976" N e Longitude -42°35'50,256" S; deste, segue confrontando com RIO AGUA BRANCA, com azimute geodésico de 271°31' e distância de 
39,57m  até o vértice DBT-P-1006101, com Latitude -14°51'21,941" N e Longitude -42°35'51,579" S; deste, segue confrontando com RIO AGUA BRANCA, com azimute geodésico de 301°01' e distância de 
48,32m  até o vértice DBT-P-1006102, com Latitude -14°51'21,131" N e Longitude -42°35'52,964" S; deste, segue confrontando com RIO AGUA BRANCA, com azimute geodésico de 308°29' e distância de 
37,21m  até o vértice DBT-P-1006103, com Latitude -14°51'20,378" N e Longitude -42°35'53,938" S; deste, segue confrontando com RIO AGUA BRANCA, com azimute geodésico de 260°24' e distância de 
39,83m  até o vértice DBT-V-1000102, com Latitude -14°51'20,593" N e Longitude -42°35'55,251" S; deste, segue confrontando com RIO AGUA BRANCA, com azimute geodésico de 310°21' e distância de 
28,85m  até o vértice DBT-V-1000103, com Latitude -14°51'19,986" N e Longitude -42°35'55,986" S; deste, segue confrontando com RIO AGUA BRANCA, com azimute geodésico de 292°52' e distância de 
58,18m  até o vértice DBT-V-1000104, com Latitude -14°51'19,250" N e Longitude -42°35'57,779" S; deste, segue confrontando com RIO AGUA BRANCA, com azimute geodésico de 283°24' e distância de 
112,54m  até o vértice DBT-V-1000105, com Latitude -14°51'18,401" N e Longitude -42°36'01,441" S; deste, segue confrontando com RIO AGUA BRANCA, com azimute geodésico de 292°53' e distância de 
83,63m  até o vértice DBT-V-1000106, com Latitude -14°51'17,342" N e Longitude -42°36'04,018" S; deste, segue confrontando com RIO AGUA BRANCA, com azimute geodésico de 269°14' e distância de 
53,86m  até o vértice DBT-V-1000107, com Latitude -14°51'17,366" N e Longitude -42°36'05,819" S; deste, segue confrontando com CNS: 01.017-3 - Mat.:882, com azimute geodésico de 25°29' e distância de 
73,13m  até o vértice DBT-M-A036, com Latitude -14°51'15,218" N e Longitude -42°36'04,767" S; deste, segue confrontando com CNS: 01.017-3 - Mat.:882, com azimute geodésico de 39°01' e distância de 
518,41m  até o vértice DBT-M-A037, com Latitude -14°51'02,117" N e Longitude -42°35'53,849" S; deste, segue confrontando com CNS: 01.017-3 - Mat.:3217, com azimute geodésico de 76°34' e distância de 
1692,03m  até o vértice DBT-M-5059 encerrando esta descrição. Todas as coordenadas aqui descritas estão georrefereciadas ao Sistema Geodésico Brasileiro, e encontram-se representadas no sistema SIRGAS 2000, referenciadas ao Meridiano Central -39°. Todos os azimutes e distâncias, área e perímetro foram calculados no Sistema Geodésico Local.Inicia-se a descrição deste perímetro no vértice DBT-M-5059, com Latitude -14°50'49,334" N e Longitude -42°34'58,808" S; deste, segue confrontando com ANAHI DO CARMO CERQUEIRA SANTANA, com azimute geodésico de 193°26' e distância de 60,27m até o vértice DBT-M-5058, com Latitude -14°50'51,241" N e Longitude -42°34'59,277" S; deste, segue confrontando com ANAHI DO CARMO CERQUEIRA SANTANA, com azimute geodésico de 183°42' e distância de 
53,91m  até o vértice DBT-M-5057, com Latitude -14°50'52,991" N e Longitude -42°34'59,393" S; deste, segue confrontando com ANAHI DO CARMO CERQUEIRA SANTANA, com azimute geodésico de 170°46' e distância de 
48,65m  até o vértice DBT-M-5056, com Latitude -14°50'54,554" N e Longitude -42°34'59,132" S; deste, segue confrontando com WELISMAR BORBOREMA CARVALHO, com azimute geodésico de 205°50' e distância de 
559,05m  até o vértice DBT-M-5172, com Latitude -14°51'10,922" N e Longitude -42°35'07,281" S; deste, segue confrontando com SEBASTIÃO SOUZA E SILVA, com azimute geodésico de 306°29' e distância de 
460,93m  até o vértice DBT-M-A053, com Latitude -14°51'02,004" N e Longitude -42°35'19,673" S; deste, segue confrontando com SEBASTIÃO SOUZA E SILVA, com azimute geodésico de 243°25' e distância de 
721,29m  até o vértice DBT-M-A052, com Latitude -14°51'12,502" N e Longitude -42°35'41,247" S; deste, segue confrontando com SEBASTIÃO SOUZA E SILVA, com azimute geodésico de 222°45' e distância de 
396,7m  até o vértice DBT-M-A051, com Latitude -14°51'21,976" N e Longitude -42°35'50,256" S; deste, segue confrontando com RIO AGUA BRANCA, com azimute geodésico de 271°31' e distância de 
39,57m  até o vértice DBT-P-1006101, com Latitude -14°51'21,941" N e Longitude -42°35'51,579" S; deste, segue confrontando com RIO AGUA BRANCA, com azimute geodésico de 301°01' e distância de 
48,32m  até o vértice DBT-P-1006102, com Latitude -14°51'21,131" N e Longitude -42°35'52,964" S; deste, segue confrontando com RIO AGUA BRANCA, com azimute geodésico de 308°29' e distância de 
37,21m  até o vértice DBT-P-1006103, com Latitude -14°51'20,378" N e Longitude -42°35'53,938" S; deste, segue confrontando com RIO AGUA BRANCA, com azimute geodésico de 260°24' e distância de 
39,83m  até o vértice DBT-V-1000102, com Latitude -14°51'20,593" N e Longitude -42°35'55,251" S; deste, segue confrontando com RIO AGUA BRANCA, com azimute geodésico de 310°21' e distância de 
28,85m  até o vértice DBT-V-1000103, com Latitude -14°51'19,986" N e Longitude -42°35'55,986" S; deste, segue confrontando com RIO AGUA BRANCA, com azimute geodésico de 292°52' e distância de 
58,18m  até o vértice DBT-V-1000104, com Latitude -14°51'19,250" N e Longitude -42°35'57,779" S; deste, segue confrontando com RIO AGUA BRANCA, com azimute geodésico de 283°24' e distância de 
112,54m  até o vértice DBT-V-1000105, com Latitude -14°51'18,401" N e Longitude -42°36'01,441" S; deste, segue confrontando com RIO AGUA BRANCA, com azimute geodésico de 292°53' e distância de 
83,63m  até o vértice DBT-V-1000106, com Latitude -14°51'17,342" N e Longitude -42°36'04,018" S; deste, segue confrontando com RIO AGUA BRANCA, com azimute geodésico de 269°14' e distância de 
53,86m  até o vértice DBT-V-1000107, com Latitude -14°51'17,366" N e Longitude -42°36'05,819" S; deste, segue confrontando com CNS: 01.017-3 - Mat.:882, com azimute geodésico de 25°29' e distância de 
73,13m  até o vértice DBT-M-A036, com Latitude -14°51'15,218" N e Longitude -42°36'04,767" S; deste, segue confrontando com CNS: 01.017-3 - Mat.:882, com azimute geodésico de 39°01' e distância de 
518,41m  até o vértice DBT-M-A037, com Latitude -14°51'02,117" N e Longitude -42°35'53,849" S; deste, segue confrontando com CNS: 01.017-3 - Mat.:3217, com azimute geodésico de 76°34' e distância de 
1692,03m  até o vértice DBT-M-5059 encerrando esta descrição. Todas as coordenadas aqui descritas estão georrefereciadas ao Sistema Geodésico Brasileiro, e encontram-se representadas no sistema SIRGAS 2000, referenciadas ao Meridiano Central -39°. Todos os azimutes e distâncias, área e perímetro foram calculados no Sistema Geodésico Local.Inicia-se a descrição deste perímetro no vértice DBT-M-5059, com Latitude -14°50'49,334" N e Longitude -42°34'58,808" S; deste, segue confrontando com ANAHI DO CARMO CERQUEIRA SANTANA, com azimute geodésico de 193°26' e distância de 60,27m até o vértice DBT-M-5058, com Latitude -14°50'51,241" N e Longitude -42°34'59,277" S; deste, segue confrontando com ANAHI DO CARMO CERQUEIRA SANTANA, com azimute geodésico de 183°42' e distância de 
53,91m  até o vértice DBT-M-5057, com Latitude -14°50'52,991" N e Longitude -42°34'59,393" S; deste, segue confrontando com ANAHI DO CARMO CERQUEIRA SANTANA, com azimute geodésico de 170°46' e distância de 
48,65m  até o vértice DBT-M-5056, com Latitude -14°50'54,554" N e Longitude -42°34'59,132" S; deste, segue confrontando com WELISMAR BORBOREMA CARVALHO, com azimute geodésico de 205°50' e distância de 
559,05m  até o vértice DBT-M-5172, com Latitude -14°51'10,922" N e Longitude -42°35'07,281" S; deste, segue confrontando com SEBASTIÃO SOUZA E SILVA, com azimute geodésico de 306°29' e distância de 
460,93m  até o vértice DBT-M-A053, com Latitude -14°51'02,004" N e Longitude -42°35'19,673" S; deste, segue confrontando com SEBASTIÃO SOUZA E SILVA, com azimute geodésico de 243°25' e distância de 
721,29m  até o vértice DBT-M-A052, com Latitude -14°51'12,502" N e Longitude -42°35'41,247" S; deste, segue confrontando com SEBASTIÃO SOUZA E SILVA, com azimute geodésico de 222°45' e distância de 
396,7m  até o vértice DBT-M-A051, com Latitude -14°51'21,976" N e Longitude -42°35'50,256" S; deste, segue confrontando com RIO AGUA BRANCA, com azimute geodésico de 271°31' e distância de 
39,57m  até o vértice DBT-P-1006101, com Latitude -14°51'21,941" N e Longitude -42°35'51,579" S; deste, segue confrontando com RIO AGUA BRANCA, com azimute geodésico de 301°01' e distância de 
48,32m  até o vértice DBT-P-1006102, com Latitude -14°51'21,131" N e Longitude -42°35'52,964" S; deste, segue confrontando com RIO AGUA BRANCA, com azimute geodésico de 308°29' e distância de 
37,21m  até o vértice DBT-P-1006103, com Latitude -14°51'20,378" N e Longitude -42°35'53,938" S; deste, segue confrontando com RIO AGUA BRANCA, com azimute geodésico de 260°24' e distância de 
39,83m  até o vértice DBT-V-1000102, com Latitude -14°51'20,593" N e Longitude -42°35'55,251" S; deste, segue confrontando com RIO AGUA BRANCA, com azimute geodésico de 310°21' e distância de 
28,85m  até o vértice DBT-V-1000103, com Latitude -14°51'19,986" N e Longitude -42°35'55,986" S; deste, segue confrontando com RIO AGUA BRANCA, com azimute geodésico de 292°52' e distância de 
58,18m  até o vértice DBT-V-1000104, com Latitude -14°51'19,250" N e Longitude -42°35'57,779" S; deste, segue confrontando com RIO AGUA BRANCA, com azimute geodésico de 283°24' e distância de 
112,54m  até o vértice DBT-V-1000105, com Latitude -14°51'18,401" N e Longitude -42°36'01,441" S; deste, segue confrontando com RIO AGUA BRANCA, com azimute geodésico de 292°53' e distância de 
83,63m  até o vértice DBT-V-1000106, com Latitude -14°51'17,342" N e Longitude -42°36'04,018" S; deste, segue confrontando com RIO AGUA BRANCA, com azimute geodésico de 269°14' e distância de 
53,86m  até o vértice DBT-V-1000107, com Latitude -14°51'17,366" N e Longitude -42°36'05,819" S; deste, segue confrontando com CNS: 01.017-3 - Mat.:882, com azimute geodésico de 25°29' e distância de 
73,13m  até o vértice DBT-M-A036, com Latitude -14°51'15,218" N e Longitude -42°36'04,767" S; deste, segue confrontando com CNS: 01.017-3 - Mat.:882, com azimute geodésico de 39°01' e distância de 
518,41m  até o vértice DBT-M-A037, com Latitude -14°51'02,117" N e Longitude -42°35'53,849" S; deste, segue confrontando com CNS: 01.017-3 - Mat.:3217, com azimute geodésico de 76°34' e distância de 
1692,03m  até o vértice DBT-M-5059 encerrando esta descrição. Todas as coordenadas aqui descritas estão georrefereciadas ao Sistema Geodésico Brasileiro, e encontram-se representadas no sistema SIRGAS 2000, referenciadas ao Meridiano Central -39°. Todos os azimutes e distâncias, área e perímetro foram calculados no Sistema Geodésico Local.Inicia-se a descrição deste perímetro no vértice DBT-M-5059, com Latitude -14°50'49,334" N e Longitude -42°34'58,808" S; deste, segue confrontando com ANAHI DO CARMO CERQUEIRA SANTANA, com azimute geodésico de 193°26' e distância de 60,27m até o vértice DBT-M-5058, com Latitude -14°50'51,241" N e Longitude -42°34'59,277" S; deste, segue confrontando com ANAHI DO CARMO CERQUEIRA SANTANA, com azimute geodésico de 183°42' e distância de 
53,91m  até o vértice DBT-M-5057, com Latitude -14°50'52,991" N e Longitude -42°34'59,393" S; deste, segue confrontando com ANAHI DO CARMO CERQUEIRA SANTANA, com azimute geodésico de 170°46' e distância de 
48,65m  até o vértice DBT-M-5056, com Latitude -14°50'54,554" N e Longitude -42°34'59,132" S; deste, segue confrontando com WELISMAR BORBOREMA CARVALHO, com azimute geodésico de 205°50' e distância de 
559,05m  até o vértice DBT-M-5172, com Latitude -14°51'10,922" N e Longitude -42°35'07,281" S; deste, segue confrontando com SEBASTIÃO SOUZA E SILVA, com azimute geodésico de 306°29' e distância de 
460,93m  até o vértice DBT-M-A053, com Latitude -14°51'02,004" N e Longitude -42°35'19,673" S; deste, segue confrontando com SEBASTIÃO SOUZA E SILVA, com azimute geodésico de 243°25' e distância de 
721,29m  até o vértice DBT-M-A052, com Latitude -14°51'12,502" N e Longitude -42°35'41,247" S; deste, segue confrontando com SEBASTIÃO SOUZA E SILVA, com azimute geodésico de 222°45' e distância de 
396,7m  até o vértice DBT-M-A051, com Latitude -14°51'21,976" N e Longitude -42°35'50,256" S; deste, segue confrontando com RIO AGUA BRANCA, com azimute geodésico de 271°31' e distância de 
39,57m  até o vértice DBT-P-1006101, com Latitude -14°51'21,941" N e Longitude -42°35'51,579" S; deste, segue confrontando com RIO AGUA BRANCA, com azimute geodésico de 301°01' e distância de 
48,32m  até o vértice DBT-P-1006102, com Latitude -14°51'21,131" N e Longitude -42°35'52,964" S; deste, segue confrontando com RIO AGUA BRANCA, com azimute geodésico de 308°29' e distância de 
37,21m  até o vértice DBT-P-1006103, com Latitude -14°51'20,378" N e Longitude -42°35'53,938" S; deste, segue confrontando com RIO AGUA BRANCA, com azimute geodésico de 260°24' e distância de 
39,83m  até o vértice DBT-V-1000102, com Latitude -14°51'20,593" N e Longitude -42°35'55,251" S; deste, segue confrontando com RIO AGUA BRANCA, com azimute geodésico de 310°21' e distância de 
28,85m  até o vértice DBT-V-1000103, com Latitude -14°51'19,986" N e Longitude -42°35'55,986" S; deste, segue confrontando com RIO AGUA BRANCA, com azimute geodésico de 292°52' e distância de 
58,18m  até o vértice DBT-V-1000104, com Latitude -14°51'19,250" N e Longitude -42°35'57,779" S; deste, segue confrontando com RIO AGUA BRANCA, com azimute geodésico de 283°24' e distância de 
112,54m  até o vértice DBT-V-1000105, com Latitude -14°51'18,401" N e Longitude -42°36'01,441" S; deste, segue confrontando com RIO AGUA BRANCA, com azimute geodésico de 292°53' e distância de 
83,63m  até o vértice DBT-V-1000106, com Latitude -14°51'17,342" N e Longitude -42°36'04,018" S; deste, segue confrontando com RIO AGUA BRANCA, com azimute geodésico de 269°14' e distância de 
53,86m  até o vértice DBT-V-1000107, com Latitude -14°51'17,366" N e Longitude -42°36'05,819" S; deste, segue confrontando com CNS: 01.017-3 - Mat.:882, com azimute geodésico de 25°29' e distância de 
73,13m  até o vértice DBT-M-A036, com Latitude -14°51'15,218" N e Longitude -42°36'04,767" S; deste, segue confrontando com CNS: 01.017-3 - Mat.:882, com azimute geodésico de 39°01' e distância de 
518,41m  até o vértice DBT-M-A037, com Latitude -14°51'02,117" N e Longitude -42°35'53,849" S; deste, segue confrontando com CNS: 01.017-3 - Mat.:3217, com azimute geodésico de 76°34' e distância de 
1692,03m  até o vértice DBT-M-5059 encerrando esta descrição. Todas as coordenadas aqui descritas estão georrefereciadas ao Sistema Geodésico Brasileiro, e encontram-se representadas no sistema SIRGAS 2000, referenciadas ao Meridiano Central -39°. Todos os azimutes e distâncias, área e perímetro foram calculados no Sistema Geodésico Local.Inicia-se a descrição deste perímetro no vértice DBT-M-5059, com Latitude -14°50'49,334" N e Longitude -42°34'58,808" S; deste, segue confrontando com ANAHI DO CARMO CERQUEIRA SANTANA, com azimute geodésico de 193°26' e distância de 60,27m até o vértice DBT-M-5058, com Latitude -14°50'51,241" N e Longitude -42°34'59,277" S; deste, segue confrontando com ANAHI DO CARMO CERQUEIRA SANTANA, com azimute geodésico de 183°42' e distância de 
53,91m  até o vértice DBT-M-5057, com Latitude -14°50'52,991" N e Longitude -42°34'59,393" S; deste, segue confrontando com ANAHI DO CARMO CERQUEIRA SANTANA, com azimute geodésico de 170°46' e distância de 
48,65m  até o vértice DBT-M-5056, com Latitude -14°50'54,554" N e Longitude -42°34'59,132" S; deste, segue confrontando com WELISMAR BORBOREMA CARVALHO, com azimute geodésico de 205°50' e distância de 
559,05m  até o vértice DBT-M-5172, com Latitude -14°51'10,922" N e Longitude -42°35'07,281" S; deste, segue confrontando com SEBASTIÃO SOUZA E SILVA, com azimute geodésico de 306°29' e distância de 
460,93m  até o vértice DBT-M-A053, com Latitude -14°51'02,004" N e Longitude -42°35'19,673" S; deste, segue confrontando com SEBASTIÃO SOUZA E SILVA, com azimute geodésico de 243°25' e distância de 
721,29m  até o vértice DBT-M-A052, com Latitude -14°51'12,502" N e Longitude -42°35'41,247" S; deste, segue confrontando com SEBASTIÃO SOUZA E SILVA, com azimute geodésico de 222°45' e distância de 
396,7m  até o vértice DBT-M-A051, com Latitude -14°51'21,976" N e Longitude -42°35'50,256" S; deste, segue confrontando com RIO AGUA BRANCA, com azimute geodésico de 271°31' e distância de 
39,57m  até o vértice DBT-P-1006101, com Latitude -14°51'21,941" N e Longitude -42°35'51,579" S; deste, segue confrontando com RIO AGUA BRANCA, com azimute geodésico de 301°01' e distância de 
48,32m  até o vértice DBT-P-1006102, com Latitude -14°51'21,131" N e Longitude -42°35'52,964" S; deste, segue confrontando com RIO AGUA BRANCA, com azimute geodésico de 308°29' e distância de 
37,21m  até o vértice DBT-P-1006103, com Latitude -14°51'20,378" N e Longitude -42°35'53,938" S; deste, segue confrontando com RIO AGUA BRANCA, com azimute geodésico de 260°24' e distância de 
39,83m  até o vértice DBT-V-1000102, com Latitude -14°51'20,593" N e Longitude -42°35'55,251" S; deste, segue confrontando com RIO AGUA BRANCA, com azimute geodésico de 310°21' e distância de 
28,85m  até o vértice DBT-V-1000103, com Latitude -14°51'19,986" N e Longitude -42°35'55,986" S; deste, segue confrontando com RIO AGUA BRANCA, com azimute geodésico de 292°52' e distância de 
58,18m  até o vértice DBT-V-1000104, com Latitude -14°51'19,250" N e Longitude -42°35'57,779" S; deste, segue confrontando com RIO AGUA BRANCA, com azimute geodésico de 283°24' e distância de 
112,54m  até o vértice DBT-V-1000105, com Latitude -14°51'18,401" N e Longitude -42°36'01,441" S; deste, segue confrontando com RIO AGUA BRANCA, com azimute geodésico de 292°53' e distância de 
83,63m  até o vértice DBT-V-1000106, com Latitude -14°51'17,342" N e Longitude -42°36'04,018" S; deste, segue confrontando com RIO AGUA BRANCA, com azimute geodésico de 269°14' e distância de 
53,86m  até o vértice DBT-V-1000107, com Latitude -14°51'17,366" N e Longitude -42°36'05,819" S; deste, segue confrontando com CNS: 01.017-3 - Mat.:882, com azimute geodésico de 25°29' e distância de 
73,13m  até o vértice DBT-M-A036, com Latitude -14°51'15,218" N e Longitude -42°36'04,767" S; deste, segue confrontando com CNS: 01.017-3 - Mat.:882, com azimute geodésico de 39°01' e distância de 
518,41m  até o vértice DBT-M-A037, com Latitude -14°51'02,117" N e Longitude -42°35'53,849" S; deste, segue confrontando com CNS: 01.017-3 - Mat.:3217, com azimute geodésico de 76°34' e distância de 
1692,03m  até o vértice DBT-M-5059 encerrando esta descrição. Todas as coordenadas aqui descritas estão georrefereciadas ao Sistema Geodésico Brasileiro, e encontram-se representadas no sistema SIRGAS 2000, referenciadas ao Meridiano Central -39°. Todos os azimutes e distâncias, área e perímetro foram calculados no Sistema Geodésico Local.Inicia-se a descrição deste perímetro no vértice DBT-M-5059, com Latitude -14°50'49,334" N e Longitude -42°34'58,808" S; deste, segue confrontando com ANAHI DO CARMO CERQUEIRA SANTANA, com azimute geodésico de 193°26' e distância de 60,27m até o vértice DBT-M-5058, com Latitude -14°50'51,241" N e Longitude -42°34'59,277" S; deste, segue confrontando com ANAHI DO CARMO CERQUEIRA SANTANA, com azimute geodésico de 183°42' e distância de 
53,91m  até o vértice DBT-M-5057, com Latitude -14°50'52,991" N e Longitude -42°34'59,393" S; deste, segue confrontando com ANAHI DO CARMO CERQUEIRA SANTANA, com azimute geodésico de 170°46' e distância de 
48,65m  até o vértice DBT-M-5056, com Latitude -14°50'54,554" N e Longitude -42°34'59,132" S; deste, segue confrontando com WELISMAR BORBOREMA CARVALHO, com azimute geodésico de 205°50' e distância de 
559,05m  até o vértice DBT-M-5172, com Latitude -14°51'10,922" N e Longitude -42°35'07,281" S; deste, segue confrontando com SEBASTIÃO SOUZA E SILVA, com azimute geodésico de 306°29' e distância de 
460,93m  até o vértice DBT-M-A053, com Latitude -14°51'02,004" N e Longitude -42°35'19,673" S; deste, segue confrontando com SEBASTIÃO SOUZA E SILVA, com azimute geodésico de 243°25' e distância de 
721,29m  até o vértice DBT-M-A052, com Latitude -14°51'12,502" N e Longitude -42°35'41,247" S; deste, segue confrontando com SEBASTIÃO SOUZA E SILVA, com azimute geodésico de 222°45' e distância de 
396,7m  até o vértice DBT-M-A051, com Latitude -14°51'21,976" N e Longitude -42°35'50,256" S; deste, segue confrontando com RIO AGUA BRANCA, com azimute geodésico de 271°31' e distância de 
39,57m  até o vértice DBT-P-1006101, com Latitude -14°51'21,941" N e Longitude -42°35'51,579" S; deste, segue confrontando com RIO AGUA BRANCA, com azimute geodésico de 301°01' e distância de 
48,32m  até o vértice DBT-P-1006102, com Latitude -14°51'21,131" N e Longitude -42°35'52,964" S; deste, segue confrontando com RIO AGUA BRANCA, com azimute geodésico de 308°29' e distância de 
37,21m  até o vértice DBT-P-1006103, com Latitude -14°51'20,378" N e Longitude -42°35'53,938" S; deste, segue confrontando com RIO AGUA BRANCA, com azimute geodésico de 260°24' e distância de 
39,83m  até o vértice DBT-V-1000102, com Latitude -14°51'20,593" N e Longitude -42°35'55,251" S; deste, segue confrontando com RIO AGUA BRANCA, com azimute geodésico de 310°21' e distância de 
28,85m  até o vértice DBT-V-1000103, com Latitude -14°51'19,986" N e Longitude -42°35'55,986" S; deste, segue confrontando com RIO AGUA BRANCA, com azimute geodésico de 292°52' e distância de 
58,18m  até o vértice DBT-V-1000104, com Latitude -14°51'19,250" N e Longitude -42°35'57,779" S; deste, segue confrontando com RIO AGUA BRANCA, com azimute geodésico de 283°24' e distância de 
112,54m  até o vértice DBT-V-1000105, com Latitude -14°51'18,401" N e Longitude -42°36'01,441" S; deste, segue confrontando com RIO AGUA BRANCA, com azimute geodésico de 292°53' e distância de 
83,63m  até o vértice DBT-V-1000106, com Latitude -14°51'17,342" N e Longitude -42°36'04,018" S; deste, segue confrontando com RIO AGUA BRANCA, com azimute geodésico de 269°14' e distância de 
53,86m  até o vértice DBT-V-1000107, com Latitude -14°51'17,366" N e Longitude -42°36'05,819" S; deste, segue confrontando com CNS: 01.017-3 - Mat.:882, com azimute geodésico de 25°29' e distância de 
73,13m  até o vértice DBT-M-A036, com Latitude -14°51'15,218" N e Longitude -42°36'04,767" S; deste, segue confrontando com CNS: 01.017-3 - Mat.:882, com azimute geodésico de 39°01' e distância de 
518,41m  até o vértice DBT-M-A037, com Latitude -14°51'02,117" N e Longitude -42°35'53,849" S; deste, segue confrontando com CNS: 01.017-3 - Mat.:3217, com azimute geodésico de 76°34' e distância de 
1692,03m  até o vértice DBT-M-5059 encerrando esta descrição. Todas as coordenadas aqui descritas estão georrefereciadas ao Sistema Geodésico Brasileiro, e encontram-se representadas no sistema SIRGAS 2000, referenciadas ao Meridiano Central -39°. Todos os azimutes e distâncias, área e perímetro foram calculados no Sistema Geodésico Local.Inicia-se a descrição deste perímetro no vértice DBT-M-5059, com Latitude -14°50'49,334" N e Longitude -42°34'58,808" S; deste, segue confrontando com ANAHI DO CARMO CERQUEIRA SANTANA, com azimute geodésico de 193°26' e distância de 60,27m até o vértice DBT-M-5058, com Latitude -14°50'51,241" N e Longitude -42°34'59,277" S; deste, segue confrontando com ANAHI DO CARMO CERQUEIRA SANTANA, com azimute geodésico de 183°42' e distância de 
53,91m  até o vértice DBT-M-5057, com Latitude -14°50'52,991" N e Longitude -42°34'59,393" S; deste, segue confrontando com ANAHI DO CARMO CERQUEIRA SANTANA, com azimute geodésico de 170°46' e distância de 
48,65m  até o vértice DBT-M-5056, com Latitude -14°50'54,554" N e Longitude -42°34'59,132" S; deste, segue confrontando com WELISMAR BORBOREMA CARVALHO, com azimute geodésico de 205°50' e distância de 
559,05m  até o vértice DBT-M-5172, com Latitude -14°51'10,922" N e Longitude -42°35'07,281" S; deste, segue confrontando com SEBASTIÃO SOUZA E SILVA, com azimute geodésico de 306°29' e distância de 
460,93m  até o vértice DBT-M-A053, com Latitude -14°51'02,004" N e Longitude -42°35'19,673" S; deste, segue confrontando com SEBASTIÃO SOUZA E SILVA, com azimute geodésico de 243°25' e distância de 
721,29m  até o vértice DBT-M-A052, com Latitude -14°51'12,502" N e Longitude -42°35'41,247" S; deste, segue confrontando com SEBASTIÃO SOUZA E SILVA, com azimute geodésico de 222°45' e distância de 
396,7m  até o vértice DBT-M-A051, com Latitude -14°51'21,976" N e Longitude -42°35'50,256" S; deste, segue confrontando com RIO AGUA BRANCA, com azimute geodésico de 271°31' e distância de 
39,57m  até o vértice DBT-P-1006101, com Latitude -14°51'21,941" N e Longitude -42°35'51,579" S; deste, segue confrontando com RIO AGUA BRANCA, com azimute geodésico de 301°01' e distância de 
48,32m  até o vértice DBT-P-1006102, com Latitude -14°51'21,131" N e Longitude -42°35'52,964" S; deste, segue confrontando com RIO AGUA BRANCA, com azimute geodésico de 308°29' e distância de 
37,21m  até o vértice DBT-P-1006103, com Latitude -14°51'20,378" N e Longitude -42°35'53,938" S; deste, segue confrontando com RIO AGUA BRANCA, com azimute geodésico de 260°24' e distância de 
39,83m  até o vértice DBT-V-1000102, com Latitude -14°51'20,593" N e Longitude -42°35'55,251" S; deste, segue confrontando com RIO AGUA BRANCA, com azimute geodésico de 310°21' e distância de 
28,85m  até o vértice DBT-V-1000103, com Latitude -14°51'19,986" N e Longitude -42°35'55,986" S; deste, segue confrontando com RIO AGUA BRANCA, com azimute geodésico de 292°52' e distância de 
58,18m  até o vértice DBT-V-1000104, com Latitude -14°51'19,250" N e Longitude -42°35'57,779" S; deste, segue confrontando com RIO AGUA BRANCA, com azimute geodésico de 283°24' e distância de 
112,54m  até o vértice DBT-V-1000105, com Latitude -14°51'18,401" N e Longitude -42°36'01,441" S; deste, segue confrontando com RIO AGUA BRANCA, com azimute geodésico de 292°53' e distância de 
83,63m  até o vértice DBT-V-1000106, com Latitude -14°51'17,342" N e Longitude -42°36'04,018" S; deste, segue confrontando com RIO AGUA BRANCA, com azimute geodésico de 269°14' e distância de 
53,86m  até o vértice DBT-V-1000107, com Latitude -14°51'17,366" N e Longitude -42°36'05,819" S; deste, segue confrontando com CNS: 01.017-3 - Mat.:882, com azimute geodésico de 25°29' e distância de 
73,13m  até o vértice DBT-M-A036, com Latitude -14°51'15,218" N e Longitude -42°36'04,767" S; deste, segue confrontando com CNS: 01.017-3 - Mat.:882, com azimute geodésico de 39°01' e distância de 
518,41m  até o vértice DBT-M-A037, com Latitude -14°51'02,117" N e Longitude -42°35'53,849" S; deste, segue confrontando com CNS: 01.017-3 - Mat.:3217, com azimute geodésico de 76°34' e distância de 
1692,03m  até o vértice DBT-M-5059 encerrando esta descrição. Todas as coordenadas aqui descritas estão georrefereciadas ao Sistema Geodésico Brasileiro, e encontram-se representadas no sistema SIRGAS 2000, referenciadas ao Meridiano Central -39°. Todos os azimutes e distâncias, área e perímetro foram calculados no Sistema Geodésico Local.Inicia-se a descrição deste perímetro no vértice DBT-M-5059, com Latitude -14°50'49,334" N e Longitude -42°34'58,808" S; deste, segue confrontando com ANAHI DO CARMO CERQUEIRA SANTANA, com azimute geodésico de 193°26' e distância de 60,27m até o vértice DBT-M-5058, com Latitude -14°50'51,241" N e Longitude -42°34'59,277" S; deste, segue confrontando com ANAHI DO CARMO CERQUEIRA SANTANA, com azimute geodésico de 183°42' e distância de 
53,91m  até o vértice DBT-M-5057, com Latitude -14°50'52,991" N e Longitude -42°34'59,393" S; deste, segue confrontando com ANAHI DO CARMO CERQUEIRA SANTANA, com azimute geodésico de 170°46' e distância de 
48,65m  até o vértice DBT-M-5056, com Latitude -14°50'54,554" N e Longitude -42°34'59,132" S; deste, segue confrontando com WELISMAR BORBOREMA CARVALHO, com azimute geodésico de 205°50' e distância de 
559,05m  até o vértice DBT-M-5172, com Latitude -14°51'10,922" N e Longitude -42°35'07,281" S; deste, segue confrontando com SEBASTIÃO SOUZA E SILVA, com azimute geodésico de 306°29' e distância de 
460,93m  até o vértice DBT-M-A053, com Latitude -14°51'02,004" N e Longitude -42°35'19,673" S; deste, segue confrontando com SEBASTIÃO SOUZA E SILVA, com azimute geodésico de 243°25' e distância de 
721,29m  até o vértice DBT-M-A052, com Latitude -14°51'12,502" N e Longitude -42°35'41,247" S; deste, segue confrontando com SEBASTIÃO SOUZA E SILVA, com azimute geodésico de 222°45' e distância de 
396,7m  até o vértice DBT-M-A051, com Latitude -14°51'21,976" N e Longitude -42°35'50,256" S; deste, segue confrontando com RIO AGUA BRANCA, com azimute geodésico de 271°31' e distância de 
39,57m  até o vértice DBT-P-1006101, com Latitude -14°51'21,941" N e Longitude -42°35'51,579" S; deste, segue confrontando com RIO AGUA BRANCA, com azimute geodésico de 301°01' e distância de 
48,32m  até o vértice DBT-P-1006102, com Latitude -14°51'21,131" N e Longitude -42°35'52,964" S; deste, segue confrontando com RIO AGUA BRANCA, com azimute geodésico de 308°29' e distância de 
37,21m  até o vértice DBT-P-1006103, com Latitude -14°51'20,378" N e Longitude -42°35'53,938" S; deste, segue confrontando com RIO AGUA BRANCA, com azimute geodésico de 260°24' e distância de 
39,83m  até o vértice DBT-V-1000102, com Latitude -14°51'20,593" N e Longitude -42°35'55,251" S; deste, segue confrontando com RIO AGUA BRANCA, com azimute geodésico de 310°21' e distância de 
28,85m  até o vértice DBT-V-1000103, com Latitude -14°51'19,986" N e Longitude -42°35'55,986" S; deste, segue confrontando com RIO AGUA BRANCA, com azimute geodésico de 292°52' e distância de 
58,18m  até o vértice DBT-V-1000104, com Latitude -14°51'19,250" N e Longitude -42°35'57,779" S; deste, segue confrontando com RIO AGUA BRANCA, com azimute geodésico de 283°24' e distância de 
112,54m  até o vértice DBT-V-1000105, com Latitude -14°51'18,401" N e Longitude -42°36'01,441" S; deste, segue confrontando com RIO AGUA BRANCA, com azimute geodésico de 292°53' e distância de 
83,63m  até o vértice DBT-V-1000106, com Latitude -14°51'17,342" N e Longitude -42°36'04,018" S; deste, segue confrontando com RIO AGUA BRANCA, com azimute geodésico de 269°14' e distância de 
53,86m  até o vértice DBT-V-1000107, com Latitude -14°51'17,366" N e Longitude -42°36'05,819" S; deste, segue confrontando com CNS: 01.017-3 - Mat.:882, com azimute geodésico de 25°29' e distância de 
73,13m  até o vértice DBT-M-A036, com Latitude -14°51'15,218" N e Longitude -42°36'04,767" S; deste, segue confrontando com CNS: 01.017-3 - Mat.:882, com azimute geodésico de 39°01' e distância de 
518,41m  até o vértice DBT-M-A037, com Latitude -14°51'02,117" N e Longitude -42°35'53,849" S; deste, segue confrontando com CNS: 01.017-3 - Mat.:3217, com azimute geodésico de 76°34' e distância de 
1692,03m  até o vértice DBT-M-5059 encerrando esta descrição. Todas as coordenadas aqui descritas estão georrefereciadas ao Sistema Geodésico Brasileiro, e encontram-se representadas no sistema SIRGAS 2000, referenciadas ao Meridiano Central -39°. Todos os azimutes e distâncias, área e perímetro foram calculados no Sistema Geodésico Local.Inicia-se a descrição deste perímetro no vértice DBT-M-5059, com Latitude -14°50'49,334" N e Longitude -42°34'58,808" S; deste, segue confrontando com ANAHI DO CARMO CERQUEIRA SANTANA, com azimute geodésico de 193°26' e distância de 60,27m até o vértice DBT-M-5058, com Latitude -14°50'51,241" N e Longitude -42°34'59,277" S; deste, segue confrontando com ANAHI DO CARMO CERQUEIRA SANTANA, com azimute geodésico de 183°42' e distância de 
53,91m  até o vértice DBT-M-5057, com Latitude -14°50'52,991" N e Longitude -42°34'59,393" S; deste, segue confrontando com ANAHI DO CARMO CERQUEIRA SANTANA, com azimute geodésico de 170°46' e distância de 
48,65m  até o vértice DBT-M-5056, com Latitude -14°50'54,554" N e Longitude -42°34'59,132" S; deste, segue confrontando com WELISMAR BORBOREMA CARVALHO, com azimute geodésico de 205°50' e distância de 
559,05m  até o vértice DBT-M-5172, com Latitude -14°51'10,922" N e Longitude -42°35'07,281" S; deste, segue confrontando com SEBASTIÃO SOUZA E SILVA, com azimute geodésico de 306°29' e distância de 
460,93m  até o vértice DBT-M-A053, com Latitude -14°51'02,004" N e Longitude -42°35'19,673" S; deste, segue confrontando com SEBASTIÃO SOUZA E SILVA, com azimute geodésico de 243°25' e distância de 
721,29m  até o vértice DBT-M-A052, com Latitude -14°51'12,502" N e Longitude -42°35'41,247" S; deste, segue confrontando com SEBASTIÃO SOUZA E SILVA, com azimute geodésico de 222°45' e distância de 
396,7m  até o vértice DBT-M-A051, com Latitude -14°51'21,976" N e Longitude -42°35'50,256" S; deste, segue confrontando com RIO AGUA BRANCA, com azimute geodésico de 271°31' e distância de 
39,57m  até o vértice DBT-P-1006101, com Latitude -14°51'21,941" N e Longitude -42°35'51,579" S; deste, segue confrontando com RIO AGUA BRANCA, com azimute geodésico de 301°01' e distância de 
48,32m  até o vértice DBT-P-1006102, com Latitude -14°51'21,131" N e Longitude -42°35'52,964" S; deste, segue confrontando com RIO AGUA BRANCA, com azimute geodésico de 308°29' e distância de 
37,21m  até o vértice DBT-P-1006103, com Latitude -14°51'20,378" N e Longitude -42°35'53,938" S; deste, segue confrontando com RIO AGUA BRANCA, com azimute geodésico de 260°24' e distância de 
39,83m  até o vértice DBT-V-1000102, com Latitude -14°51'20,593" N e Longitude -42°35'55,251" S; deste, segue confrontando com RIO AGUA BRANCA, com azimute geodésico de 310°21' e distância de 
28,85m  até o vértice DBT-V-1000103, com Latitude -14°51'19,986" N e Longitude -42°35'55,986" S; deste, segue confrontando com RIO AGUA BRANCA, com azimute geodésico de 292°52' e distância de 
58,18m  até o vértice DBT-V-1000104, com Latitude -14°51'19,250" N e Longitude -42°35'57,779" S; deste, segue confrontando com RIO AGUA BRANCA, com azimute geodésico de 283°24' e distância de 
112,54m  até o vértice DBT-V-1000105, com Latitude -14°51'18,401" N e Longitude -42°36'01,441" S; deste, segue confrontando com RIO AGUA BRANCA, com azimute geodésico de 292°53' e distância de 
83,63m  até o vértice DBT-V-1000106, com Latitude -14°51'17,342" N e Longitude -42°36'04,018" S; deste, segue confrontando com RIO AGUA BRANCA, com azimute geodésico de 269°14' e distância de 
53,86m  até o vértice DBT-V-1000107, com Latitude -14°51'17,366" N e Longitude -42°36'05,819" S; deste, segue confrontando com CNS: 01.017-3 - Mat.:882, com azimute geodésico de 25°29' e distância de 
73,13m  até o vértice DBT-M-A036, com Latitude -14°51'15,218" N e Longitude -42°36'04,767" S; deste, segue confrontando com CNS: 01.017-3 - Mat.:882, com azimute geodésico de 39°01' e distância de 
518,41m  até o vértice DBT-M-A037, com Latitude -14°51'02,117" N e Longitude -42°35'53,849" S; deste, segue confrontando com CNS: 01.017-3 - Mat.:3217, com azimute geodésico de 76°34' e distância de 
1692,03m  até o vértice DBT-M-5059 encerrando esta descrição. Todas as coordenadas aqui descritas estão georrefereciadas ao Sistema Geodésico Brasileiro, e encontram-se representadas no sistema SIRGAS 2000, referenciadas ao Meridiano Central -39°. Todos os azimutes e distâncias, área e perímetro foram calculados no Sistema Geodésico Local.Inicia-se a descrição deste perímetro no vértice DBT-M-5059, com Latitude -14°50'49,334" N e Longitude -42°34'58,808" S; deste, segue confrontando com ANAHI DO CARMO CERQUEIRA SANTANA, com azimute geodésico de 193°26' e distância de 60,27m até o vértice DBT-M-5058, com Latitude -14°50'51,241" N e Longitude -42°34'59,277" S; deste, segue confrontando com ANAHI DO CARMO CERQUEIRA SANTANA, com azimute geodésico de 183°42' e distância de 
53,91m  até o vértice DBT-M-5057, com Latitude -14°50'52,991" N e Longitude -42°34'59,393" S; deste, segue confrontando com ANAHI DO CARMO CERQUEIRA SANTANA, com azimute geodésico de 170°46' e distância de 
48,65m  até o vértice DBT-M-5056, com Latitude -14°50'54,554" N e Longitude -42°34'59,132" S; deste, segue confrontando com WELISMAR BORBOREMA CARVALHO, com azimute geodésico de 205°50' e distância de 
559,05m  até o vértice DBT-M-5172, com Latitude -14°51'10,922" N e Longitude -42°35'07,281" S; deste, segue confrontando com SEBASTIÃO SOUZA E SILVA, com azimute geodésico de 306°29' e distância de 
460,93m  até o vértice DBT-M-A053, com Latitude -14°51'02,004" N e Longitude -42°35'19,673" S; deste, segue confrontando com SEBASTIÃO SOUZA E SILVA, com azimute geodésico de 243°25' e distância de 
721,29m  até o vértice DBT-M-A052, com Latitude -14°51'12,502" N e Longitude -42°35'41,247" S; deste, segue confrontando com SEBASTIÃO SOUZA E SILVA, com azimute geodésico de 222°45' e distância de 
396,7m  até o vértice DBT-M-A051, com Latitude -14°51'21,976" N e Longitude -42°35'50,256" S; deste, segue confrontando com RIO AGUA BRANCA, com azimute geodésico de 271°31' e distância de 
39,57m  até o vértice DBT-P-1006101, com Latitude -14°51'21,941" N e Longitude -42°35'51,579" S; deste, segue confrontando com RIO AGUA BRANCA, com azimute geodésico de 301°01' e distância de 
48,32m  até o vértice DBT-P-1006102, com Latitude -14°51'21,131" N e Longitude -42°35'52,964" S; deste, segue confrontando com RIO AGUA BRANCA, com azimute geodésico de 308°29' e distância de 
37,21m  até o vértice DBT-P-1006103, com Latitude -14°51'20,378" N e Longitude -42°35'53,938" S; deste, segue confrontando com RIO AGUA BRANCA, com azimute geodésico de 260°24' e distância de 
39,83m  até o vértice DBT-V-1000102, com Latitude -14°51'20,593" N e Longitude -42°35'55,251" S; deste, segue confrontando com RIO AGUA BRANCA, com azimute geodésico de 310°21' e distância de 
28,85m  até o vértice DBT-V-1000103, com Latitude -14°51'19,986" N e Longitude -42°35'55,986" S; deste, segue confrontando com RIO AGUA BRANCA, com azimute geodésico de 292°52' e distância de 
58,18m  até o vértice DBT-V-1000104, com Latitude -14°51'19,250" N e Longitude -42°35'57,779" S; deste, segue confrontando com RIO AGUA BRANCA, com azimute geodésico de 283°24' e distância de 
112,54m  até o vértice DBT-V-1000105, com Latitude -14°51'18,401" N e Longitude -42°36'01,441" S; deste, segue confrontando com RIO AGUA BRANCA, com azimute geodésico de 292°53' e distância de 
83,63m  até o vértice DBT-V-1000106, com Latitude -14°51'17,342" N e Longitude -42°36'04,018" S; deste, segue confrontando com RIO AGUA BRANCA, com azimute geodésico de 269°14' e distância de 
53,86m  até o vértice DBT-V-1000107, com Latitude -14°51'17,366" N e Longitude -42°36'05,819" S; deste, segue confrontando com CNS: 01.017-3 - Mat.:882, com azimute geodésico de 25°29' e distância de 
73,13m  até o vértice DBT-M-A036, com Latitude -14°51'15,218" N e Longitude -42°36'04,767" S; deste, segue confrontando com CNS: 01.017-3 - Mat.:882, com azimute geodésico de 39°01' e distância de 
518,41m  até o vértice DBT-M-A037, com Latitude -14°51'02,117" N e Longitude -42°35'53,849" S; deste, segue confrontando com CNS: 01.017-3 - Mat.:3217, com azimute geodésico de 76°34' e distância de 
1692,03m  até o vértice DBT-M-5059 encerrando esta descrição. Todas as coordenadas aqui descritas estão georrefereciadas ao Sistema Geodésico Brasileiro, e encontram-se representadas no sistema SIRGAS 2000, referenciadas ao Meridiano Central -39°. Todos os azimutes e distâncias, área e perímetro foram calculados no Sistema Geodésico Local.Inicia-se a descrição deste perímetro no vértice DBT-M-5059, com Latitude -14°50'49,334" N e Longitude -42°34'58,808" S; deste, segue confrontando com ANAHI DO CARMO CERQUEIRA SANTANA, com azimute geodésico de 193°26' e distância de 60,27m até o vértice DBT-M-5058, com Latitude -14°50'51,241" N e Longitude -42°34'59,277" S; deste, segue confrontando com ANAHI DO CARMO CERQUEIRA SANTANA, com azimute geodésico de 183°42' e distância de 
53,91m  até o vértice DBT-M-5057, com Latitude -14°50'52,991" N e Longitude -42°34'59,393" S; deste, segue confrontando com ANAHI DO CARMO CERQUEIRA SANTANA, com azimute geodésico de 170°46' e distância de 
48,65m  até o vértice DBT-M-5056, com Latitude -14°50'54,554" N e Longitude -42°34'59,132" S; deste, segue confrontando com WELISMAR BORBOREMA CARVALHO, com azimute geodésico de 205°50' e distância de 
559,05m  até o vértice DBT-M-5172, com Latitude -14°51'10,922" N e Longitude -42°35'07,281" S; deste, segue confrontando com SEBASTIÃO SOUZA E SILVA, com azimute geodésico de 306°29' e distância de 
460,93m  até o vértice DBT-M-A053, com Latitude -14°51'02,004" N e Longitude -42°35'19,673" S; deste, segue confrontando com SEBASTIÃO SOUZA E SILVA, com azimute geodésico de 243°25' e distância de 
721,29m  até o vértice DBT-M-A052, com Latitude -14°51'12,502" N e Longitude -42°35'41,247" S; deste, segue confrontando com SEBASTIÃO SOUZA E SILVA, com azimute geodésico de 222°45' e distância de 
396,7m  até o vértice DBT-M-A051, com Latitude -14°51'21,976" N e Longitude -42°35'50,256" S; deste, segue confrontando com RIO AGUA BRANCA, com azimute geodésico de 271°31' e distância de 
39,57m  até o vértice DBT-P-1006101, com Latitude -14°51'21,941" N e Longitude -42°35'51,579" S; deste, segue confrontando com RIO AGUA BRANCA, com azimute geodésico de 301°01' e distância de 
48,32m  até o vértice DBT-P-1006102, com Latitude -14°51'21,131" N e Longitude -42°35'52,964" S; deste, segue confrontando com RIO AGUA BRANCA, com azimute geodésico de 308°29' e distância de 
37,21m  até o vértice DBT-P-1006103, com Latitude -14°51'20,378" N e Longitude -42°35'53,938" S; deste, segue confrontando com RIO AGUA BRANCA, com azimute geodésico de 260°24' e distância de 
39,83m  até o vértice DBT-V-1000102, com Latitude -14°51'20,593" N e Longitude -42°35'55,251" S; deste, segue confrontando com RIO AGUA BRANCA, com azimute geodésico de 310°21' e distância de 
28,85m  até o vértice DBT-V-1000103, com Latitude -14°51'19,986" N e Longitude -42°35'55,986" S; deste, segue confrontando com RIO AGUA BRANCA, com azimute geodésico de 292°52' e distância de 
58,18m  até o vértice DBT-V-1000104, com Latitude -14°51'19,250" N e Longitude -42°35'57,779" S; deste, segue confrontando com RIO AGUA BRANCA, com azimute geodésico de 283°24' e distância de 
112,54m  até o vértice DBT-V-1000105, com Latitude -14°51'18,401" N e Longitude -42°36'01,441" S; deste, segue confrontando com RIO AGUA BRANCA, com azimute geodésico de 292°53' e distância de 
83,63m  até o vértice DBT-V-1000106, com Latitude -14°51'17,342" N e Longitude -42°36'04,018" S; deste, segue confrontando com RIO AGUA BRANCA, com azimute geodésico de 269°14' e distância de 
53,86m  até o vértice DBT-V-1000107, com Latitude -14°51'17,366" N e Longitude -42°36'05,819" S; deste, segue confrontando com CNS: 01.017-3 - Mat.:882, com azimute geodésico de 25°29' e distância de 
73,13m  até o vértice DBT-M-A036, com Latitude -14°51'15,218" N e Longitude -42°36'04,767" S; deste, segue confrontando com CNS: 01.017-3 - Mat.:882, com azimute geodésico de 39°01' e distância de 
518,41m  até o vértice DBT-M-A037, com Latitude -14°51'02,117" N e Longitude -42°35'53,849" S; deste, segue confrontando com CNS: 01.017-3 - Mat.:3217, com azimute geodésico de 76°34' e distância de 
1692,03m  até o vértice DBT-M-5059 encerrando esta descrição. Todas as coordenadas aqui descritas estão georrefereciadas ao Sistema Geodésico Brasileiro, e encontram-se representadas no sistema SIRGAS 2000, referenciadas ao Meridiano Central -39°. Todos os azimutes e distâncias, área e perímetro foram calculados no Sistema Geodésico Local.
>>>>>>> b2c0ce20a453c9b521425215c24f3dd665009c2a
