\documentclass[10.8pt, a4paper]{article}
%ABNT
\usepackage[utf8]{inputenc}
\usepackage[lmargin=3cm, tmargin=3cm, rmargin=2cm, bmargin=2cm]{geometry}%configurando as margens
\usepackage[onehalfspacing]{setspace}
\usepackage[T1]{fontenc}
\usepackage[brazil]{babel}



%Pacotes essenciais
\usepackage{graphicx, xcolor, comment, enumerate, multirow, multicol, indentfirst}


%Pacotes de Matemática
\usepackage{amsmath, amsthm, amsfonts, amssymb, dsfont, mathtools, blindtext}

%Outros
\usepackage{lipsum}
\usepackage{fancyhdr}%Para fazer cabeçalhos personalizados
%\usepackage[grid]{eso-pic}%cria um grid na página

%Configurações do Cabeçalho
%\fancyhf{}%linha horizontal
%\fancyhead[R]{\small{\textsl{Relatório de Análise de Mercado de Terras - RAMT\\Superintendência Regional da Bahia/SR-05}}}
%\pagestyle{fancy}
%\lhead{\includegraphics[width=6.5cm, height=0.8cm]{Logo_incra.jpeg}}

% Papel Timbrado
%\usepackage{background}
%\usepackage{pgfplots}
%\backgroundsetup{contents={\includegraphics{timbre.png}}, scale=1, placement=top, opacity=1}
%pacotes e configurações


\begin{document}

\begin{center}
\LARGE {MEMORIAL DESCRITIVO}
\end{center}
\begin{tabular}{ll}
\emph{Imóvel: FAZENDA BARRO VERMELHO} & \emph{Comarca: (01.017-3) Urandi - BA} \\ 
\emph{Proprietário: JOSÉ APARECIDO SANTOS}&\\ 
\emph{UF: Ba}& \emph{Município: Urandi}\\ 
\emph{Código INCRA: 000.019.168.386-2}\hspace{3cm}	 & \emph{Matrícula: 3211}\\ 
\emph{Área ($ha$): 76,8515 ha} & \emph{Perímetro ($m$): 5.086,36 m}\\ 
\end{tabular}
\vspace{1cm}
% corpo do memorial

Inicia-se a descrição deste perímetro no vértice P-1, de coordenadas N 8609714.003 m e E 551898.964 m, Datum SIRGAS 2000  com Meridiano Central -39, localizado a , Código INCRA 951.137.639.770-7;  deste, segue confrontando com Faz Cobezinho Srª. Sônia, com os seguintes azimute plano e distância:161°43'2.47'' e 99.50; até o vértice P-2, de coordenadas N 8609619.528 m e E 551930.177 m; deste, segue confrontando com Faz Cobezinho Srª. Sônia, com os seguintes azimute plano e distância:167°19'12.36'' e 289.73; até o vértice P-3, de coordenadas N 8609336.867 m e E 551993.773 m; deste, segue confrontando com Estrada Vicinal, com os seguintes azimute plano e distância:228°15'14.32'' e 34.53; até o vértice P-4, de coordenadas N 8609313.875 m e E 551968.009 m; deste, segue confrontando com Faz Cobezinho (mat. 1159) Valdélio Souza Gayoso Sá Barreto, com os seguintes azimute plano e distância:320°45'42.50'' e 427.62; até o vértice P-5, de coordenadas N 8609645.078 m e E 551697.518 m; deste, segue confrontando com Faz Cobezinho - Reinaldo Souza Gayoso Sá Barreto / Valdélio Souza Gayoso Sá Barreto, com os seguintes azimute plano e distância:71°06'42.06'' e 212.91; até o vértice P-1, de coordenadas N 8609714.003 m e E 551898.964 m, encerrando esta descrição. Todas as coordenadas aqui descritas estão georrefereciadas ao Sistema Geodésico Brasileiro, e encontram-se representadas no sistema UTM, referenciadas ao Meridiano Central -39, tendo como DATUM SIRGAS 2000 .Todos os azimutes e distâncias, área e perímetro foram calculados no plano de projeção UTM.
\vspace{1cm}
\begin{flushright}
	\emph{São Sebastião do Passé, 17 de outubro de 2021}
\end{flushright}
\vspace{0.5cm}
\begin{flushleft}
\begin{tabular}{l}
\emph{Alessandre Gabriel Oliveira Ramos}\\
\emph{Engº Agrônomo - RNP: 0500668892}\\
\emph{Esp. em Georreferenciamento de Imóveis Rurais}\\
\emph{MSc. em Fitotecnia} \\	
\end{tabular}
\end{flushleft}
\end{document}          
